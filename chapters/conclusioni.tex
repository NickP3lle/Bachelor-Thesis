\chapter{Considerazioni finali}
\label{cap:conclusioni}

\section{Risultati ottenuti}

Nonostante gli obiettivi iniziali non siano stati definiti in modo chiaro e siano stati quindi modificati durante il corso dello stage, il lavoro svolto è stato completato con successo e ha prodotto buoni risultati.
Infatti, l'implementazione originale della tecnologia generatrice di token \emph{API} firmati con certificati creati da chiavi ellittiche ed autenticati con metodologia \emph{HMAC} non è stata portata a termine.
Tuttavia, è stata sostituita con l'implementazione del \emph{ChatBot} per Microsoft Teams.

\noindent Tra i prodotti finali del tirocinio possiamo trovare:
\begin{itemize}
	\item Un documento che riassume il funzionamento dei token \emph{JWT} e degli algoritmi utilizzati per la creazione della firma digitale e della loro trasmissione sicura.
	\item Un \emph{ChatBot} per Microsoft Teams in grado di gestire diverse richieste contemporaneamente in totale sicurezza.
	\item Diversi piccoli script in \emph{Python} realizzati per contruibuire alle esigenze dell'azienda e per facilitare il lavoro dei colleghi.
\end{itemize}

\section{Competenze acquisite}

Durante il tirocinio ho avuto l'opportunità di approfondire le mie conoscenze e di acquisirne di nuove.

Questa esperienza mi ha permesso di comprendere meglio il funzionamento di un'azienda e di come viene organizzato il lavoro al suo interno, sia individualmente che in gruppo.

Inoltre, ho approfondito le mie conoscenze in ambito di sicurezza informatica.
Ho scoperto che i processi di autenticazione, generalemente trasparenti per l'utente, sono molto più complessi di quanto si possa immaginare, ma sono fondamentali per garantire la privacy.

Infine, lavorare con nuove tecnologie, come \emph{Microsoft Teams} e \emph{Microsoft Graph}, e nuovi linguaggi di programmazione, come \emph{C\#}, ha ampliato il mio bagaglio di competenze informatiche.

\section{Valutazione personale}

Credo fermamente che l'esperienza di stage sia stata molto positiva e formativa.
Mi ha permesso di crescere sia professionalmente che personalmente, grazie all'ambiente aziendale dinamico e alla collaborazione con colleghi esperti e disponibili.

Sono convinto che le conoscenze acquisite nell'ambito della sicurezza informatica mi saranno molto utili in futuro, in quanto è un campo in continua evoluzione in cui è necessario avere una formazione di base solida.
Grazie alla creazione del \emph{ChatBot} ho potuto mettere in pratica queste conoscenze e apprendere un nuovo linguaggio di programmazione.

Infine, ritengo che il lavoro svolto abbia portato a buoni risultati e che il \emph{ChatBot} realizzato possa essere un valido strumento per l'azienda.

Il buon rapporto con i colleghi e la soddisfazione dell'azienda per la collaborazione degli scorsi mesi mi spingono a pensare che il mio contributo sia stato apprezzato e che il lavoro svolto sia stato di qualità.