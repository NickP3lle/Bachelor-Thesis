\chapter{Introduzione}
\label{cap:introduzione}


\section{L'azienda}

Lo stage è stato svolto presso la sede operativa di Prorob S.r.l. in Grisignano di Zocco (VI).

\begin{figure}[!ht] 
    \centering 
    \includegraphics[width=0.6\columnwidth]{introduzione/prorob-Logo} 
    \caption{Logo di Prorob S.r.l.}
\end{figure}

Prorob è una startup fondata nel 2017 come avventura di tre ingegneri meccatronici laureati presso l'Università di Padova.
I primi progetti dell'azienda si basano su competenze ingegneristiche e capacità di ricerca e sviluppo di alto livello.

Nella sua prima fase, Prorob si concentra su consulenze su misura e collaborazioni ad hoc, attraverso attività che spaziano dalla robotica all'automatizzazione industriale.
Questo ha portato a disegnare il primo prodotto, ProNet-IoT, che connette una vasta gamma di device in una piattaforma digitale per la gestione e il controllo di processi produttivi.

Con ProNet-IoT, Prorob inizia il percorso di passaggio da azienda \emph{project-based} ad azienda \emph{product-based}.
Inoltre, grazie all'ingresso di Considi S.p.A., società esperta in consulenza nel settore manufatturiero, e ad un team più grande e diversificato, Prorob evolve il proprio prodotto con l'obiettivo di sbloccare il potere dei dati di produzione attraverso l'intelligenza artificiale. Nasce così Quindi.

\bigskip
\vfill
\hfill

\subsection{Quindi}
Quindi è il primo \emph{\gls{AI} Production Copilot}.

\bigskip

\begin{figure}[!ht] 
    \centering 
    \includegraphics[width=0.6\columnwidth]{introduzione/quindi-Logo} 
    \caption{Logo di Quindi}
\end{figure}

Quindi è una soluzione \emph{\gls{SaaS}} che aiuta le aziende manifatturiere a estrarre tutto il valore dai dati della loro produzione.

Quindi integra rapidamente i dati della produzione: macchine, materiali, ordini, persone, energia. Spesso, senza alcun bisogno di input umani. In questo modo il dato è certo e ognuno può dedicarsi davvero al proprio ruolo.

Elaborando questi dati, l'\emph{AI} del \emph{Copilot} Quindi affianca i manager per prevedere scenari produttivi e pianificare la produzione, per monitorare l’avanzamento dei flussi produttivi e per rispondere in tempo reale ad ogni evento critico.

Grazie a Quindi, le aziende riducono gli sprechi di tempo e di energia, migliorando la capacità di previsione, pianificazione e risposta alle anomalie e costruendo un ambiente proattivo e inclusivo per ogni collaboratore. E \emph{quindi}, ottengono processi produttivi più efficienti e sostenibili.


\section{Contesto}

Ogni giorno condividiamo una vasta quantità di informazioni personali e confidenziali attraverso vari servizi web, dai social network alle banche.
Queste informazioni, se non adeguatamente protette dai vari metodi di autenticazione, possono diventare bersagli facili di attacchi informatici.
Infatti ogni accesso a un account, ogni transazione online e ogni semplice login contengono dati che, se caduti nelle mani sbagliate, potrebbero causare danni significativi.

Da diversi anni, l'accesso a risorse online protette è diventato un processo molto frequente per la maggior parte delle persone e spesso non ci rendiamo conto di quanto sia importante proteggere le nostre credenziali di accesso.
Una password debole o riutilizzata può diventare l'elemento che un malintenzionato potrebbe sfruttare per accedere a informazioni sensibili.
E non è solo una questione di password: l'intero processo di autenticazione deve essere sicuro.

Questi metodi, oltre a verificare l'identità di un utente prima di consentire l'accesso alle risorse protette, devono anche garantire che i dati trasmessi durante il processo di autenticazione rimangano confidenziali e integri.
Ma quali sono i protocolli e le tecnologie che rendono tutto questo possibile? Come possiamo essere sicuri che le nostre informazioni siano davvero protette quando accediamo a un servizio online?

Lo stage si propone di realizzare uno studio in grado di rispondere a queste domande, analizzando l'autenticazione attraverso \emph{JSON Web Token (JWT)} e i protocolli necessari per garantire la completa sicurezza del processo di autenticazione.


\section{Organizzazione del documento}
\label{sec:organizzazione-documento}

\begin{description}
    \item[{\hyperref[cap:inquadramento-stage]{Il secondo capitolo}}] introduce lo stage, descrivendo il concetto di autenticazione e presentando l'idea del progetto.
    
    \item[{\hyperref[cap:autenticazione-jwt]{Il terzo capitolo}}] approfondisce l'utilizzo dell'autenticazione basata su \emph{\gls{token}} di tipo \emph{JWT} e i protocolli necessari per garantire la loro sicurezza.
    
    \item[{\hyperref[cap:chatbot]{Il quarto capitolo}}] descrive l'implementazione del \emph{ChatBot}, con particolare attenzione all'utilizzo di algoritmi e tecnologie utilizzate per garantire la completa protezione dei dati scambiati nelle trasmissioni. 
    
    \item[{\hyperref[cap:conclusioni]{Il quinto capitolo}}] riguarda le considerazioni finali e la conclusione dello stage.
\end{description}

Riguardo la stesura del documento sono state adottate le seguenti convenzioni tipografiche:
\begin{itemize}
	\item gli acronimi, le abbreviazioni e i termini ambigui o di uso non comune menzionati vengono definiti nel glossario, situato alla fine del presente documento;
	\item per la prima occorrenza dei termini riportati nel glossario viene utilizzata la seguente nomenclatura: \emph{\hyperref[sec:organizzazione-documento]{parola}};
	\item i termini in lingua straniera o facenti parti del gergo tecnico sono evidenziati con il carattere \emph{corsivo}.
\end{itemize}
