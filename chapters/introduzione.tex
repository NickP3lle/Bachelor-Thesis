\chapter{Introduzione}
\label{cap:introduzione}


\section{L'azienda}

Lo stage è stato svolto presso la sede operativa di Prorob S.r.l. in Grisignano di Zocco (VI).


\section{Contesto}

Ogni giorno condividiamo una vasta quantità di informazioni personali e confidenziali attraverso vari servizi web, dai social network alle banche.
Queste informazioni, se non adeguatamente protette dai vari metodi di autenticazione, possono diventare bersagli facili di attacchi informatici.
Infatti ogni accesso a un account, ogni transazione online e ogni semplice login contengono dati che, se caduti nelle mani sbagliate, potrebbero causare danni significativi.

Da diversi anni, l'accesso a risorse online protette è diventato un processo molto frequente per la maggior parte delle persone e spesso non ci rendiamo conto di quanto sia importante proteggere le nostre credenziali di accesso.
Una password debole o riutilizzata può diventare l'elemento che un malintenzionato potrebbe sfruttare per accedere a informazioni sensibili.
E non è solo una questione di password: l'intero processo di autenticazione deve essere sicuro.

Questi metodi, oltre a verificare l'identità di un utente prima di consentire l'accesso alle risorse protette, devono anche garantire che i dati trasmessi durante il processo di autenticazione rimangano confidenziali e integri.
Ma quali sono i protocolli e le tecnologie che rendono tutto questo possibile? Come possiamo essere sicuri che le nostre informazioni siano davvero protette quando accediamo a un servizio online?

Lo stage si propone di realizzare uno studio in grado di rispondere a queste domande, analizzando l'autenticazione attraverso \emph{JSON Web Token (JWT)} e i protocolli necessari per garantire la completa sicurezza del processo di autenticazione.


\section{Organizzazione del documento}

\begin{description}
    \item[{\hyperref[cap:inquadramento-stage]{Il secondo capitolo}}] introduce lo stage, descrivendo il concetto di autenticazione e presentando l'idea del progetto.
    
    \item[{\hyperref[cap:autenticazione-jwt]{Il terzo capitolo}}] approfondisce l'utilizzo dell'autenticazione basata su \emph{\gls{token}}\glsfirstoccur di tipo \emph{JWT} e i protocolli necessari per garantire la loro sicurezza.
    
    \item[{\hyperref[cap:analisi-requisiti]{Il quarto capitolo}}] approfondisce ...
    
    \item[{\hyperref[cap:progettazione-codifica]{Il quinto capitolo}}] approfondisce ...
\end{description}

Riguardo la stesura del documento sono state adottate le seguenti convenzioni tipografiche:
\begin{itemize}
	\item gli acronimi, le abbreviazioni e i termini ambigui o di uso non comune menzionati vengono definiti nel glossario, situato alla fine del presente documento;
	\item per la prima occorrenza dei termini riportati nel glossario viene utilizzata la seguente nomenclatura: \emph{parola}\glsfirstoccur;
	\item i termini in lingua straniera o facenti parti del gergo tecnico sono evidenziati con il carattere \emph{corsivo}.
\end{itemize}
