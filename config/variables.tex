\newcommand{\myName}{Nicolò Pellegrinelli}
\newcommand{\myID}{2034334}
\newcommand{\myTitle}{Studio sull'Autenticazione Sicura nell'Ecosistema Web 2.0}
\newcommand{\myDegree}{Tesi di laurea}
\newcommand{\myUni}{Università degli Studi di Padova}
% For BSc level just use "Corso di Laurea" and don't add "Triennale" to it
\newcommand{\myFaculty}{Corso di Laurea in Informatica}
\newcommand{\myDepartment}{Dipartimento di Matematica ``Tullio Levi-Civita''}
\newcommand{\profTitle}{Prof.}
\newcommand{\myProf}{Alessandro Brighente}
\newcommand{\myLocation}{Padova}
\newcommand{\myAA}{2023-2024}
\newcommand{\myTime}{Luglio 2024}

% PDF file metadata fields
% when updating them delete the build directory, otherwise they won't change
\begin{filecontents*}{\jobname.xmpdata}
  \Title{Studio sull'Autenticazione Sicura nell'Ecosistema Web 2.0}
  \Author{Nicolò Pellegrinelli}
  \Language{it-IT}
  \Subject{L'accesso a risorse online protette deve garantire confidenzialità e sicurezza dei dati sensibili degli utenti. Attraverso un'indagine approfondita delle attuali pratiche e tecnologie di autenticazione, questo studio esplora le sfide e le opportunità nel garantire un accesso sicuro alle piattaforme web moderne. Dalla valutazione delle vulnerabilità esistenti alla progettazione e implementazione di soluzioni innovative, questa ricerca si propone di fornire una panoramica esauriente sulle migliori pratiche per l'autenticazione sicura nel contesto del Web 2.0}
  \Keywords{Autenticazione\sep Sicurezza\sep Crittografia\sep Firma Digitale\sep Token\sep JSON Web Token\sep JWT}
\end{filecontents*}
