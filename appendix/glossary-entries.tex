% Acronyms
% \newacronym[description={\glslink{apig}{Application Program Interface}}]
%     {api}{API}{Application Program Interface}

% \newacronym[description={\glslink{jwtg}{JSON Web Token}}]
%     {jwt}{JWT}{JSON Web Token}

% Glossary entries
\newglossaryentry{api} {
    % name=\glslink{api}{API},
    name=Application Program Interface,
    text=Application Program Interface,
    sort=api,
    description={in informatica con il termine \emph{Application Programming Interface} (ing. interfaccia di programmazione di un'applicazione) si indica ogni insieme di procedure disponibili al programmatore, di solito raggruppate a formare un set di strumenti specifici per l'espletamento di un determinato compito all'interno di un certo programma. La finalità è ottenere un'astrazione, di solito tra l'hardware e il programmatore o tra software a basso e quello ad alto livello semplificando così il lavoro di programmazione}
}

\newglossaryentry{jwt} {
    % name=\glslink{jwt}{JWT},
    name=JSON Web Token,
    text=JWT,
    sort=jwt,
    description={un \emph{JSON Web Token} (ing. stringa di tipo \emph{JSON}) è un metodo per rappresentare in modo compatto informazioni tra due parti. Le informazioni possono essere verificate e fidate perché sono firmate digitalmente. Un \emph{JWT} è composto da tre parti: un'intestazione, un contenuto e una firma. Il contenuto contiene le informazioni che si vogliono trasmettere, mentre la firma viene utilizzata per verificare che il mittente del \emph{token} sia chi dice di essere e per garantire che il contenuto del \emph{token} non sia stato alterato}
}

\newglossaryentry{token} {
    % name=\glslink{token}{Token},
    name=Token,
    text=token,
    sort=token,
    description={un \emph{token} è un oggetto simbolico, generalmente una stringa di caratteri, che rappresenta un insieme di permessi o dati di autenticazione. Viene generalmente utilizzato per verificare l'identità di un utente o di un'applicazione durante le comunicazioni con un server.}
}

\newglossaryentry{web cookie} {
    name=Web cookie,
    text=web cookie,
    sort=web cookie,
    description={un \emph{cookie} è un piccolo blocco di testo che i siti web inviano al browser e vengono memorizzati sul dispositivo dell'utente. I \emph{cookie} vengono utilizzati per memorizzare informazioni sulle visite e le impostazioni del sito web, come la lingua preferita e altre impostazioni. Questo può rendere la visita successiva più facile e il sito più utile}
}