% Glossary entries
\newglossaryentry{JWT} {
    % name=\glslink{jwt}{JWT},
    name=JWT,
    text=JSON Web Token (JWT),
    sort=jwt,
    description={un \emph{JSON Web Token} (ing. stringa di tipo \emph{JSON}) è un metodo per rappresentare in modo compatto informazioni tra due parti. Le informazioni possono essere verificate e fidate perché sono firmate digitalmente. Un \emph{JWT} è composto da tre parti: un'intestazione, un contenuto e una firma. Il contenuto contiene le informazioni che si vogliono trasmettere, mentre la firma viene utilizzata per verificare che il mittente del \emph{token} sia chi dice di essere e per garantire che il contenuto del \emph{token} non sia stato alterato}
}

\newglossaryentry{token} {
    % name=\glslink{token}{Token},
    name=Token,
    text=token,
    sort=token,
    description={un \emph{token} è un oggetto simbolico, generalmente una stringa di caratteri, che rappresenta un insieme di permessi o dati di autenticazione. Viene generalmente utilizzato per verificare l'identità di un utente o di un'applicazione durante le comunicazioni con un server}
}

\newglossaryentry{API} {
    % name=\glslink{api}{API},
    name=API,
    text=Application Programming Interface (API),
    sort=api,
    description={una \emph{Application Programming Interface} (ing. interfaccia di programmazione di un'applicazione) è un insieme di regole e protocolli che permette a diverse applicazioni software di comunicare tra loro. Le \emph{API} definiscono i metodi e i dati che le applicazioni possono utilizzare per richiedere servizi e scambiare informazioni, facilitando l'interoperabilità e l'integrazione tra sistemi diversi}
}

\newglossaryentry{web cookie} {
    name=Web cookie,
    text=web cookie,
    sort=web cookie,
    description={un \emph{cookie} è un piccolo blocco di testo che i siti web inviano al browser e vengono memorizzati sul dispositivo dell'utente. I \emph{cookie} vengono utilizzati per memorizzare informazioni sulle visite e le impostazioni del sito web, come la lingua preferita e altre impostazioni. Questo può rendere la visita successiva più facile e il sito più utile}
}

\newglossaryentry{API RESTful} {
    name=API RESTful,
    text=API RESTful,
    sort=api restful,
    description={le \emph{API RESTful} sono una categoria di \emph{API} che rispetta i vincoli dell'architettura \emph{REST}. Questi vincoli includono l'uso di un'interfaccia uniforme, la separazione client-server e la statelessness (assenza di stato), il che significa che ogni richiesta dal client al server deve contenere tutte le informazioni necessarie per comprendere e processare la richiesta. Le \emph{API RESTful} sono progettate per essere efficienti, scalabili e facili da mantenere}
}

\newglossaryentry{IoT} {
    name=IoT,
    text=Internet of Things (IoT),
    sort=iot,
    description={l'\emph{Internet of Things} (ing. Internet delle cose) è un concetto che si riferisce alla connessione di dispositivi fisici, come elettrodomestici, veicoli e sensori, alla rete Internet. Questi dispositivi possono comunicare tra loro e con altri sistemi, raccogliendo e scambiando dati per automatizzare processi e migliorare l'efficienza e la qualità della vita}
}

\newglossaryentry{JSON} {
    name=JSON,
    text=JSON,
    sort=json,
    description={\emph{JavaScript Object Notation} (ing. notazione di un oggetto \emph{JavaScript}) è un formato di scambio dati leggero e indipendente dal linguaggio di programmazione. È basato su un sottoinsieme del linguaggio di programmazione JavaScript e viene utilizzato per rappresentare oggetti e dati in modo comprensibile ma allo stesso tempo facile da interpretare per i computer}
}

\newglossaryentry{HMAC} {
    name=HMAC,
    text=HMAC,
    sort=hmac,
    description={\emph{Hash-based Message Authentication Code} (ing. codice di autenticazione del messaggio basato su \emph{hash}) è una tecnica di autenticazione basata su funzioni di \emph{hash crittografiche}. Un \emph{HMAC} consente a entrambe le parti di una comunicazione di verificare l'integrità e l'autenticità dei dati scambiati, utilizzando una chiave segreta condivisa}
}

\newglossaryentry{RSA} {
    name=RSA,
    text=RSA,
    sort=rsa,
    description={\emph{RSA} è un algoritmo di crittografia asimmetrica basato su due chiavi, una pubblica e una privata. Il nome RSA deriva dai cognomi dei suoi inventori, Ron Rivest, Adi Shamir e Leonard Adleman}
}

\newglossaryentry{ECC} {
    name=ECC,
    text=ECC,
    sort=ecc,
    description={\emph{Elliptic Curve Cryptography} (ing. crittografia a curve ellittiche) è una tecnica di crittografia asimmetrica basata su curve ellittiche algebriche. Le curve ellittiche sono utilizzate per generare coppie di chiavi pubbliche e private, che possono essere utilizzate per crittografare e decrittografare i dati}
}

\newglossaryentry{SSO} {
    name=SSO,
    text=Singe Sign-On (SSO),
    sort=sso,
    description={\emph{Single Sign-On} (ing. accesso singolo) è un metodo di autenticazione che consente a un utente di accedere a più applicazioni o servizi con un'unica identità digitale. L'utente effettua l'accesso una sola volta e può accedere a tutte le risorse autorizzate senza dover effettuare l'accesso ripetutamente}
}

\newglossaryentry{Base64} {
    name=Base64,
    text=Base64,
    sort=base64,
    description={\emph{Base64} è una tecnica di codifica che consente di rappresentare dati binari in una forma leggibile e trasferibile tramite testo. La codifica \emph{Base64} converte i dati binari in una sequenza di caratteri \emph{ASCII}, che possono essere facilmente trasferiti tramite protocolli di comunicazione testuale}
}

\newglossaryentry{URL} {
    name=URL,
    text=URL,
    sort=url,
    description={\emph{Uniform Resource Locator} (ing. localizzatore uniforme di risorse) è l'indirizzo di un documento o di una risorsa su Internet. Un \emph{URL} specifica il protocollo di comunicazione da utilizzare, il nome del server, il percorso della risorsa e altri parametri necessari per accedere alla risorsa}
}

\newglossaryentry{JWE} {
    name=JWE,
    text=JSON Web Encryption (JWE),
    sort=jwe,
    description={\emph{JSON Web Encryption} (ing. crittografia di un oggetto \emph{JSON}) è uno standard che definisce un modo per crittografare i contenuti di un \emph{JSON Web Token}. Il \emph{JWE} consente di proteggere i dati sensibili contenuti in un \emph{JWT} utilizzando algoritmi di crittografia simmetrica o asimmetrica}
}

\newglossaryentry{JWS} {
    name=JWS,
    text=JSON Web Signature (JWS),
    sort=jws,
    description={\emph{JSON Web Signature} (ing. firma di un oggetto \emph{JSON}) è uno standard che definisce un modo per firmare digitalmente un \emph{JSON Web Token}. Il \emph{JWS} consente di verificare l'autenticità e l'integrità dei dati contenuti in un \emph{JWT} utilizzando algoritmi di firma crittografica}
}

\newglossaryentry{JWA} {
    name=JWA,
    text=JSON Web Algorithms (JWA),
    sort=jwa,
    description={\emph{JSON Web Algorithms} (ing. algoritmi di un oggetto \emph{JSON}) è uno standard che definisce gli algoritmi crittografici utilizzati per firmare e crittografare i contenuti di un \emph{JSON Web Token}. Il \emph{JWA} specifica i tipi di algoritmi supportati e le modalità di utilizzo}
}

\newglossaryentry{SSA} {
    name=SSA,
    text=Signature Scheme with Appendix (SSA),
    sort=ssa,
    description={\emph{Signature Scheme with Appendix} (ing. schema di firma con appendice) è un tipo di schema di firma digitale che produce una firma separata dai dati da firmare. La firma viene allegata ai dati originali e può essere verificata utilizzando la chiave pubblica del firmatario}
}

\newglossaryentry{RNG} {
    name=RNG,
    text=Random Number Generator (RNG),
    sort=rng,
    description={un \emph{Random Number Generator} (ing. generatore di numeri casuali) è un algoritmo che genera sequenze di numeri casuali. I numeri generati sono utilizzati in crittografia, simulazioni, giochi e in molti altri contesti in cui è necessario un elemento di casualità}
}

\newglossaryentry{GNFS} {
    name=GNFS,
    text=General Number Field Sieve (GNFS),
    sort=gnfs,
    description={la \emph{General Number Field Sieve} (ing. setaccio generale del campo numerico) è un algoritmo per la fattorizzazione di numeri interi. È il miglior algoritmo conosciuto per fattorizzare numeri interi di grandi dimensioni e viene utilizzato per rompere le chiavi crittografiche basate sulla fattorizzazione di numeri primi}
}

\newglossaryentry{QS} {
    name=QS,
    text=Quadratic Sieve (QS),
    sort=qs,
    description={la \emph{Quadratic Sieve} (ing. setaccio quadratico) è un algoritmo per la fattorizzazione di numeri interi. È un algoritmo efficiente per fattorizzare numeri interi di medie dimensioni e viene utilizzato per rompere le chiavi crittografiche basate sulla fattorizzazione di numeri composti}
}

\newglossaryentry{SHA-256} {
    name=SHA-256,
    text=SHA-256,
    sort=sha-256,
    description={\emph{Secure Hash Algorithm 256-bit} (ing. algoritmo di \emph{hash} sicuro a 256 bit) è una funzione di \emph{hash crittografica} che produce un valore di \emph{hash} di 256 bit. La funzione \emph{SHA-256} è utilizzata per generare \emph{hash} di dati e messaggi, che possono essere utilizzati per verificare l'integrità e l'autenticità dei dati}
}

\newglossaryentry{TLS} {
    name=TLS,
    text=Transport Layer Security (TLS),
    sort=tls,
    description={\emph{Transport Layer Security} (ing. sicurezza del livello di trasporto) è un protocollo di sicurezza che garantisce la privacy e l'integrità delle comunicazioni su Internet. Il \emph{TLS} crittografa i dati scambiati tra client e server, proteggendoli da intercettazioni e manipolazioni}
}

\newglossaryentry{HTTPS} {
    name=HTTPS,
    text=HTTPS,
    sort=https,
    description={\emph{HyperText Transfer Protocol Secure} (ing. protocollo di trasferimento di ipertesto sicuro) è una versione sicura del protocollo \emph{HTTP} utilizzata per la comunicazione sicura su Internet. \emph{HTTPS} utilizza il protocollo \emph{TLS} per crittografare i dati scambiati tra client e server, garantendo la privacy e l'integrità delle comunicazioni}
}

\newglossaryentry{PKCS1} {
    name=PKCS\#1,
    text=PKCS\#1,
    sort=pkcs1,
    description={\emph{Public-Key Cryptography Standards \#1} (ing. standard di crittografia a chiave pubblica \#1) è uno standard che definisce i formati di chiavi pubbliche e private, le firme digitali e gli algoritmi di crittografia asimmetrica. Il \emph{PKCS\#1} è utilizzato per la generazione e la gestione delle chiavi crittografiche}
}

\newglossaryentry{P-256} {
    name=P-256,
    text=P-256,
    sort=p-256,
    description={la curva ellittica \emph{P-256} è una curva ellittica a 256 bit utilizzata per la crittografia asimmetrica. La curva \emph{P-256} è stata definita dallo standard \emph{NIST} ed è ampiamente utilizzata per la generazione di chiavi crittografiche}
}

\newglossaryentry{MITM} {
    name=MITM,
    text=Man-in-the-Middle (MITM),
    sort=mitm,
    description={nell'attacco \emph{Man-in-the-Middle} (ing. uomo nel mezzo) un aggressore intercetta e manipola le comunicazioni tra due parti senza che nessuna delle due parti lo sappia. L'attaccante può intercettare, modificare o iniettare dati nelle comunicazioni, compromettendo la privacy e l'integrità dei dati}
}

\newglossaryentry{Impersonation Attack} {
    name=Impersonation Attack,
    text=Impersonation Attack,
    sort=impersonation attack,
    description={un attacco di \emph{impersonation} (ing. impersonificazione) è un tipo di attacco in cui un aggressore si finge di essere un'altra persona o entità per ottenere accesso non autorizzato a informazioni o risorse}
}