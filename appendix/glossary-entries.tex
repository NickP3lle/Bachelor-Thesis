% Acronyms
% \newacronym[description={\glslink{apig}{Application Program Interface}}]
%     {api}{API}{Application Program Interface}

% \newacronym[description={\glslink{jwtg}{JSON Web Token}}]
%     {jwt}{JWT}{JSON Web Token}

% Glossary entries
\newglossaryentry{JWT} {
    % name=\glslink{jwt}{JWT},
    name=JWT,
    text=JSON Web Token (JWT),
    sort=jwt,
    description={un \emph{JSON Web Token} (ing. stringa di tipo \emph{JSON}) è un metodo per rappresentare in modo compatto informazioni tra due parti. Le informazioni possono essere verificate e fidate perché sono firmate digitalmente. Un \emph{JWT} è composto da tre parti: un'intestazione, un contenuto e una firma. Il contenuto contiene le informazioni che si vogliono trasmettere, mentre la firma viene utilizzata per verificare che il mittente del \emph{token} sia chi dice di essere e per garantire che il contenuto del \emph{token} non sia stato alterato}
}

\newglossaryentry{token} {
    % name=\glslink{token}{Token},
    name=Token,
    text=token,
    sort=token,
    description={un \emph{token} è un oggetto simbolico, generalmente una stringa di caratteri, che rappresenta un insieme di permessi o dati di autenticazione. Viene generalmente utilizzato per verificare l'identità di un utente o di un'applicazione durante le comunicazioni con un server.}
}

\newglossaryentry{API} {
    % name=\glslink{api}{API},
    name=API,
    text=Application Programming Interface (API),
    sort=api,
    description={una \emph{Application Programming Interface} (ing. interfaccia di programmazione di un'applicazione) è un insieme di regole e protocolli che permette a diverse applicazioni software di comunicare tra loro. Le \emph{API} definiscono i metodi e i dati che le applicazioni possono utilizzare per richiedere servizi e scambiare informazioni, facilitando l'interoperabilità e l'integrazione tra sistemi diversi.}
}

\newglossaryentry{web cookie} {
    name=Web cookie,
    text=web cookie,
    sort=web cookie,
    description={un \emph{cookie} è un piccolo blocco di testo che i siti web inviano al browser e vengono memorizzati sul dispositivo dell'utente. I \emph{cookie} vengono utilizzati per memorizzare informazioni sulle visite e le impostazioni del sito web, come la lingua preferita e altre impostazioni. Questo può rendere la visita successiva più facile e il sito più utile}
}

\newglossaryentry{API RESTful} {
    name=API RESTful,
    text=API RESTful,
    sort=api restful,
    description={le \emph{API RESTful} sono una categoria di \emph{API} che rispetta i vincoli dell'architettura \emph{REST}. Questi vincoli includono l'uso di un'interfaccia uniforme, la separazione client-server e la statelessness (assenza di stato), il che significa che ogni richiesta dal client al server deve contenere tutte le informazioni necessarie per comprendere e processare la richiesta. Le \emph{API RESTful} sono progettate per essere efficienti, scalabili e facili da mantenere}
}

\newglossaryentry{IoT} {
    name=IoT,
    text=Internet of Things (IoT),
    sort=iot,
    description={l'\emph{Internet of Things} (ing. Internet delle cose) è un concetto che si riferisce alla connessione di dispositivi fisici, come elettrodomestici, veicoli e sensori, alla rete Internet. Questi dispositivi possono comunicare tra loro e con altri sistemi, raccogliendo e scambiando dati per automatizzare processi e migliorare l'efficienza e la qualità della vita}
}